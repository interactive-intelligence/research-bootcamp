\documentclass[letterpaper,11pt]{article}

\usepackage[shortlabels]{enumitem}
\usepackage[margin=1in]{geometry}
\usepackage[most]{tcolorbox}

\begin{document}
\title{{\bf Module 2: Subfield Selection} }
\author{Name: }

\date{}
\maketitle

\section{Intro}
Now that you have chosen a subfield, it's time to get to work! This worksheet is a guide for how to approach deep diving into one subfield as a group, from which you will create a research question and project. Make sure to spend time on it! The time you spend now reading and thoroughly understanding the field will pay dividends when you create a project -- you will spend exponentially less time following footsteps and more time creating new things. Therefore, you will be reading at least TWO papers in your selected subfield. I highly recommend reading more if you feel inspired!
\newline
\newline
There won't be many questions this week, but rather a focus on reading and understanding. A pro tip: a huge portion of your tuition is spent on resources that allow you to access the frontiers of knowledge. UW has subscriptions to most prominent journals like Nature and ScienceDirect for example (as well as the NYT and WSJ!) and the UW libraries are an absolute WEALTH of information. It's a gift (and your tuition) so let's use it!!
\section{Understanding the Field}


\begin{enumerate}
    \item First, glance at the presentation spec that you will use to make a presentation next week. You do not have to do this yet, but you should know what's coming up.
    \item 
    To begin, please find at least 3 researchers in this subfield, and list their contact information here (email, or maybe LinkedIn). One tip that you should know is that researchers LOVE to talk about their papers, and if you email them with well thought out questions, you will likely get a response. A second tip is that their emails are often listed at the top of papers they write, especially in arXiv.
    \begin{tcolorbox}
TODO: Your answer here
\newline
\newline
\newline
\newline
\newline
\newline
\newline
\newline
\newline
\newline
\newline
\end{tcolorbox}

\item 
Make a list of at least ten specific papers in the subfield here. Next, \textbf{meet with your group}, and choose at least two papers each to read. You may have multiple members read the same paper or you may all read different papers. Finally, summarize \textbf{both} papers' findings or arguments, add key takeaways and vocabulary you learned, and add questions you still have here!
\begin{tcolorbox}
TODO: Your answer here
\newline
\newline
\newline
\newline
\newline
\newline
\newline
\newline
\newline
\newline
\newline
\newline
\newline
\newline
\newline
\newline
\newline
\newline
\newline
\newline
\newline
\newline
\newline
\newline
\newline
\newline
\newline
\newline
\newline
\newline
\newline
\newline
\newline
\newline
\newline
\newline
\newline
\end{tcolorbox}

\end{enumerate}



\end{document}
